\documentclass[12pt,pdftex,a4paper]{article}
\usepackage[ngerman]{babel}
\usepackage{amsmath}
\usepackage{amssymb}
\usepackage{booktabs}
\usepackage{bbm}
\usepackage{amsmath}
\newcommand{\bbN}{\mathbbm{N}}
\newcommand{\bbR}{\mathbbm{R}}
\newcommand{\bbZ}{\mathbbm{Z}}
\newcommand{\bbI}{\mathbbm{I}}
\usepackage[pdftex]{graphicx}
\usepackage{listings}
\lstset{language=Python,basicstyle=\footnotesize}
\title{Computergrafik Übungsblatt 2}

\author{Claudio Stanullo (2942506}
\date{\today}

\begin {document}
\maketitle
\section*{Aufgabe 1}
\subsection*{1.}
\textbf{Warum sind nachts alle Katzen grau?}
\subsection*{2.}
\textbf{Was versteht man unter additiver und subtraktiver Farbmischung und wo treten diese auf?}
\subsection*{3.}
\textbf{Bestimmen Sie die entsprechende Darstellung der RGB-Farbe (1.0, 0.0, 0.0) (Wertebereich [0, 1] für alle Farbkanäle) in den Farbmodellen CMYK, HSV und HSL (HSV-Farbton $\boldsymbol{0^\circ}$). Geben Sie den Rechenweg an.}
\subsection*{4.}
\textbf{Warum ist gelbe Schrift (reines Gelb) auf weißem Hintergrund eine ungeeignete Kombination? Begründen Sie Ihre Antwort mit Hilfe der Luminanzgleichung. Geben Sie Ihren Rechenweg an.}

\end {document}