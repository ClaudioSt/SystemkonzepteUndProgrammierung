\documentclass[12pt,pdftex,a4paper]{article}
\usepackage[ngerman]{babel}
\usepackage{amsmath}
\usepackage{amssymb}
\usepackage{booktabs}
\usepackage{bbm}
\newcommand{\bbN}{\mathbbm{N}}
\newcommand{\bbR}{\mathbbm{R}}
\newcommand{\bbZ}{\mathbbm{Z}}
\newcommand{\bbI}{\mathbbm{I}}
\usepackage[pdftex]{graphicx}
\usepackage{listings}
\lstset{language=Python,basicstyle=\footnotesize}
\title{Systemkonzepte und -programmierung  Aufgabenblatt 1}

\author{Claudio Stanullo (2942506), Léon Lazar (3348154), Tobias Bahr (2945684)}
\date{\today}

\begin {document}
\maketitle
\section*{Aufgabe 1: Kinokarten-Problem}
\subsection*{a)}
In der zweiten Spalte stehen nur Nullen, da alle Plätze noch leer sind.

\texttt{1 0}

\subsection*{e)}


\texttt{i. Ja. Jeder Schalter iteriert über die 100 Sitze. In Summe können also nicht mehr als 200 Plätze vergeben können..}\newline\newline
\texttt{ii. Nein. Obwohl es nur 100 zu reservierende Sitze gibt, wurden bei diesen Test in Summe 102 reserviert.}\newline\newline
\texttt{iii. Ja. Da über die Anzahl der Sitze iteriert wird und in die Datei nach der Reservierung die 1 geschrieben wird, werden alle Sitze mindestens einmal vergeben (Sofern Datei zurückgesetzt wurde). }\newline\newline
\texttt{iv. Ja. Eine Anfrage wird nur abgelehnt, wenn in der Datei beim jeweiligen Sitz bereits eine 1 steht, dieser Platz also bereits tatsächlich vergeben wurde.}\newline\newline

\section*{Aufgabe 2: }

\subsection*{a)}
\subsubsection*{i}
Registermaschine:\newline
\begin{tabular}{ll} 
\toprule
\multicolumn {2}{c }{Programm $\pi$} \\
\midrule
\multicolumn {2}{c}{Integer n $\leftarrow$ 6}\\
\multicolumn {2}{c}{Integer m $\leftarrow$ 2}\\
\midrule
\multicolumn {1}{c}{p}&\multicolumn {1}{c}{q}\\
\midrule
p1: LOAD R1, n & q1: LOAD R1, n\\
p2: LOAD R2, m & q2: LOAD R2, m\\
p3: Mul R1, R2 & q3: SUB R1, R2\\
p4: STORE R1, n & q4: STORE R1, m\\
\bottomrule
\end{tabular}
\newline\newline\newline
Stapelmaschine:\newline
\begin{tabular}{ll} 
\toprule
\multicolumn {2}{c }{Programm $\pi$} \\
\midrule
\multicolumn {2}{c}{Integer n $\leftarrow$ 6}\\
\multicolumn {2}{c}{Integer m $\leftarrow$ 2}\\
\midrule
\multicolumn {1}{c}{p}&\multicolumn {1}{c}{q}\\
\midrule
p1: PUSH n & q1: PUSH n\\
p2: PUSH m & q2: PUSH m\\
p3: MUL & q3: SUB\\
p4: POP n & q4: POP m\\
\bottomrule
\end{tabular}
\newline
\subsubsection*{ii}
Registermaschine:\newline
$p1 \rightarrow q1 \rightarrow p2 \rightarrow q2 \rightarrow p3 \rightarrow p4 \rightarrow q4$\newline
Stapelmasching:\newline
$p1 \rightarrow q1 \rightarrow p2 \rightarrow q2 \rightarrow p3 \rightarrow p4 \rightarrow q4$\newline\newline
\subsubsection*{iii}
POP. Bzw. p4, q4. Denn dies ändert die Eingangsparameter für den jeweils anderen Prozess.\newline
\subsubsection*{iv}
8 Mögliche Plätze für p1 -p4 und q1 -q4 \newline
Daraus ergeben sich 3 Bedingungen für die Reihenfolge: \newline
1) q3,q4 müssen vor p2 kommen\newline
2) vor q3 muss q1 und q2 kommen\newline
3) p1 kann vor q3 kommen \newline\newline
Es folgt:\newline\newline
Die Abläufe:\newline
q1,q2,p1,q3,...\newline
q1,p1,q2,q3,...\newline
p1,q1,q2,q3,...\newline\newline
Also 3 Möglichkeiten.\newline
Wenn jedoch alle qi's ausgeführt wurden und nur p2 -p4 übrig bleiben. Gibt es 	insgesamt 3 Möglichkeiten\newline

\subsection*{b)}
 
\subsubsection*{i}
4: $p1 \rightarrow q1 \rightarrow q2 \rightarrow q3 \rightarrow q4 \rightarrow p2 \rightarrow p3 \rightarrow p4$\newline
3: $p1 \rightarrow q1 \rightarrow p2 \rightarrow q2 \rightarrow p3 \rightarrow q3 \rightarrow p4 \rightarrow q4$\newline
2: $p1 \rightarrow q1 \rightarrow p2 \rightarrow q2 \rightarrow p3 \rightarrow q3 \rightarrow q4 \rightarrow p4$\newline\newline
Ja, es kann das Ergebnis 6 geben;\newline\newline
$q1 \rightarrow q2 \rightarrow q3 \rightarrow q4 \rightarrow p1 \rightarrow p2 \rightarrow p3 \rightarrow p4$\newline


\subsubsection*{i}
 Ändert sich nicht ggü. i. Bei i kann man q1 und q2 gleichzeitig ausführen, ohne dass es eine Auswirkung hat, nur die Platzierung von q1/q2 ist entscheidend.\newline
 

\subsubsection*{i}
 3: $q1/q2/q3 \rightarrow p1/p2/p3 \rightarrow p4 \rightarrow q4$\newline
2: $q1/q2/q3 \rightarrow p1/p2/p3 \rightarrow q4 \rightarrow p4$\newline


\end {document}