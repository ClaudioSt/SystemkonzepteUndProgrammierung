\documentclass[12pt,pdftex,a4paper]{article}
\usepackage[ngerman]{babel}
\usepackage{amsmath}
\usepackage{amssymb}
\usepackage{booktabs}
\usepackage{bbm}
\newcommand{\bbN}{\mathbbm{N}}
\newcommand{\bbR}{\mathbbm{R}}
\newcommand{\bbZ}{\mathbbm{Z}}
\newcommand{\bbI}{\mathbbm{I}}
\usepackage[pdftex]{graphicx}
\usepackage{listings}
\lstset{language=Python,basicstyle=\footnotesize}
\title{Systemkonzepte und -programmierung  Aufgabenblatt 1}

\author{Claudio Stanullo 2942506, ...}
\date{\today}

\begin {document}
\maketitle
\section*{Aufgabe 2: Schwache Fairness}
\subsection*{a)}
\textbf{Geben Sie ein Szenario an, in welchem das Programm terminiert. Beachten
Sie dabei die in der Vorlesung vorgestellte Repräsentation eines Szenarios.}\newline
\begin{tabular}[c]{|l|l|l|l|}
\hline
Prozess p&Prozess q&n&flag\\
\hline
p1: while flag is true do&\textbf{q1: while n $>$ 0 do}&1&true\\
\textbf{p1: while flag is true do} &q3: flag $\leftarrow$ false&1&true\\
p2: n $\leftarrow$ -1 * n &\textbf{q3: flag $\leftarrow$ false}&1&true\\
\textbf{p2: n $\leftarrow$ -1 * n}&q1: while n $>$ 0 do&1&false\\
p1: while flag is true do&\textbf{q1: while n $>$ 0 do}&-1&false\\
\textbf{p1: while flag is true do}&q4: (halt)&-1&false\\
p3: (halt)&q4: (halt)&-1&false\\
\hline
\end{tabular}
\subsection*{b)}
\textbf{Angenommen, wir fordern keine (schwache) Fairness für die Ausführung.
Beschreiben Sie ein Szenario, welches möglichst wenige unterschiedliche Zeilen ausführt,
und bei welchem das Programm nicht terminiert.
}\newline
\subsection*{c)}
\textbf{ Angenommen, es ist eine schwach faire Ausführung garantiert. Ist dann
eine Terminierung garantiert? Begründen Sie.}\newline
\end {document}